%%%%%%%%%%%%%%%%%%%%%%%%%%%%%%%%%%%%%%%%%%%%%%%%%%%%%%%%%%%%%%%%%%%%%%%%%%%%%%%%
% ch01_solutions.tex
% Answer Key for Chapter 1: Electron-Integral View of Quantum Chemistry
%
% Course: 2302638 Advanced Quantum Chemistry
% Institution: Department of Chemistry, Faculty of Science, Chulalongkorn University
%%%%%%%%%%%%%%%%%%%%%%%%%%%%%%%%%%%%%%%%%%%%%%%%%%%%%%%%%%%%%%%%%%%%%%%%%%%%%%%%

\documentclass[11pt,a4paper]{article}
\usepackage[margin=1in]{geometry}
\usepackage{../../solutions_style}

\title{\textbf{Chapter 1: Answer Key}\\
\large Electron-Integral View of Quantum Chemistry\\
\normalsize 2302638 Advanced Quantum Chemistry}
\author{Department of Chemistry, Chulalongkorn University}
\date{}

\begin{document}
\maketitle
\tableofcontents

%%%%%%%%%%%%%%%%%%%%%%%%%%%%%%%%%%%%%%%%%%%%%%%%%%%%%%%%%%%%%%%%%%%%%%%%%%%%%%%%
\section{Checkpoint Question Answers}
%%%%%%%%%%%%%%%%%%%%%%%%%%%%%%%%%%%%%%%%%%%%%%%%%%%%%%%%%%%%%%%%%%%%%%%%%%%%%%%%

This section provides detailed answers to all checkpoint questions from Chapter 1,
organized by their location in the chapter.

%-------------------------------------------------------------------------------
\subsection{Checkpoint 1.1: Electron Count in Nonorthonormal Basis}
\label{sec:cp11}

\begin{checkpointAnswer}[Section 1.7 -- Electron Count]
\textbf{Question:} Why is the electron count $\tr{\Pmat\Smat}$ and not $\tr{\Pmat}$ in a
nonorthonormal AO basis?

\textbf{Answer:}
The electron count formula depends on the orthonormality properties of the
basis set. Let us derive both cases to understand why the overlap matrix
$\Smat$ must appear.

\textbf{Case 1: Orthonormal MO basis.}
In an orthonormal molecular orbital (MO) basis $\{\phi_p\}$ with
$\langle \phi_p | \phi_q \rangle = \delta_{pq}$, the electron density is
\begin{equation}
    \rho(\mathbf{r}) = \sum_{p} n_p |\phi_p(\mathbf{r})|^2,
\end{equation}
where $n_p$ is the occupation number of orbital $p$ (0, 1, or 2 for RHF).
The total electron count is
\begin{equation}
    N_e = \int \rho(\mathbf{r})\, d\mathbf{r}
        = \sum_{p} n_p \underbrace{\int |\phi_p(\mathbf{r})|^2\, d\mathbf{r}}_{=1}
        = \sum_{p} n_p = \tr{\Pmat^{\text{MO}}},
\end{equation}
where $P^{\text{MO}}_{pq} = n_p \delta_{pq}$ is diagonal. Here, $\tr{\Pmat}$
gives the correct electron count because the orbitals are orthonormal.

\textbf{Case 2: Nonorthonormal AO basis.}
In the atomic orbital (AO) basis $\{\chi_\mu\}$ with overlap
$S_{\mu\nu} = \langle \chi_\mu | \chi_\nu \rangle \neq \delta_{\mu\nu}$,
the MOs are expanded as
\begin{equation}
    \phi_i(\mathbf{r}) = \sum_{\mu} C_{\mu i}\, \chi_\mu(\mathbf{r}).
\end{equation}
For closed-shell RHF with $n_{\text{occ}}$ doubly-occupied orbitals, the
electron density becomes
\begin{equation}
    \rho(\mathbf{r}) = 2\sum_{i=1}^{n_{\text{occ}}} |\phi_i(\mathbf{r})|^2
    = 2\sum_{i=1}^{n_{\text{occ}}} \sum_{\mu\nu} C_{\mu i} C_{\nu i}\,
      \chi_\mu(\mathbf{r}) \chi_\nu(\mathbf{r})
    = \sum_{\mu\nu} P_{\mu\nu}\, \chi_\mu(\mathbf{r}) \chi_\nu(\mathbf{r}),
\end{equation}
where $P_{\mu\nu} = 2\sum_{i}^{\text{occ}} C_{\mu i} C_{\nu i}$ is the AO
density matrix.

The electron count is then
\begin{equation}
    N_e = \int \rho(\mathbf{r})\, d\mathbf{r}
        = \sum_{\mu\nu} P_{\mu\nu} \underbrace{\int \chi_\mu(\mathbf{r})
          \chi_\nu(\mathbf{r})\, d\mathbf{r}}_{= S_{\mu\nu}}
        = \sum_{\mu\nu} P_{\mu\nu} S_{\nu\mu}
        = \tr{\Pmat\Smat}.
\end{equation}

\begin{keyformulabox}[Physical Interpretation]
The overlap matrix $\Smat$ acts as a \textbf{metric tensor} that properly
accounts for the non-Euclidean geometry of the AO space. Just as we need
a metric tensor in curved coordinates to compute distances, we need $\Smat$
to correctly ``count'' the electron probability density when our basis
functions overlap with each other.

If we incorrectly used $\tr{\Pmat}$ in a nonorthonormal basis, we would
overcount electrons in regions where basis functions overlap significantly.
\end{keyformulabox}

\textbf{Numerical verification for \ce{H2}/STO-3G:}

For \ce{H2} at 0.74~\AA{} in the STO-3G basis:
\begin{align}
    \Pmat &= \begin{pmatrix} 0.6025 & 0.6025 \\ 0.6025 & 0.6025 \end{pmatrix}, \\
    \Smat &= \begin{pmatrix} 1.0000 & 0.6599 \\ 0.6599 & 1.0000 \end{pmatrix}.
\end{align}

Computing the traces:
\begin{itemize}
    \item $\tr{\Pmat} = 0.6025 + 0.6025 = 1.205$ (incorrect!)
    \item $\tr{\Pmat\Smat} = \sum_{\mu\nu} P_{\mu\nu} S_{\nu\mu}
          = 0.6025(1.0) + 0.6025(0.6599) + 0.6025(0.6599) + 0.6025(1.0) = 2.000$
\end{itemize}

The correct electron count of 2 is obtained only when using $\tr{\Pmat\Smat}$.
\end{checkpointAnswer}

%-------------------------------------------------------------------------------
\subsection{Checkpoint 1.2: ERI Symmetry Identification}
\label{sec:cp12}

\begin{checkpointAnswer}[Section 1.7 -- ERI Symmetries]
\textbf{Question:} The code verifies that \texttt{eri[0,1,0,1]} equals \texttt{eri[1,0,0,1]}.
Which of the 8-fold ERI symmetries does this demonstrate?
(Hint: which indices are being swapped?)

\textbf{Answer:}
Let us first establish the index correspondence in chemist's notation:
\begin{equation}
    \texttt{eri[}\mu,\nu,\lambda,\sigma\texttt{]}
    \quad\leftrightarrow\quad
    \eri{\mu}{\nu}{\lambda}{\sigma}
    = \iint \chi_\mu(\mathbf{r}_1)\chi_\nu(\mathbf{r}_1)\,
      \frac{1}{r_{12}}\,
      \chi_\lambda(\mathbf{r}_2)\chi_\sigma(\mathbf{r}_2)\,
      d\mathbf{r}_1\, d\mathbf{r}_2.
\end{equation}

Comparing the two array elements:
\begin{align}
    \texttt{eri[0,1,0,1]} &\leftrightarrow \eri{0}{1}{0}{1}, \\
    \texttt{eri[1,0,0,1]} &\leftrightarrow \eri{1}{0}{0}{1}.
\end{align}

The indices being swapped are $\mu \leftrightarrow \nu$ (positions 0 and 1),
while $\lambda$ and $\sigma$ remain unchanged. This corresponds to the
\textbf{permutation symmetry in the first electron coordinate (bra)}:

\begin{equation}
    \boxed{\eri{\mu}{\nu}{\lambda}{\sigma} = \eri{\nu}{\mu}{\lambda}{\sigma}}
    \qquad \text{(swap first two indices)}
\end{equation}

\textbf{Physical origin:} This symmetry arises because the product
$\chi_\mu(\mathbf{r}_1)\chi_\nu(\mathbf{r}_1)$ is invariant under
exchange of $\mu$ and $\nu$---we are simply multiplying two functions
at the same point $\mathbf{r}_1$, and multiplication is commutative:
\begin{equation}
    \chi_\mu(\mathbf{r}_1)\chi_\nu(\mathbf{r}_1)
    = \chi_\nu(\mathbf{r}_1)\chi_\mu(\mathbf{r}_1).
\end{equation}

\begin{keyformulabox}[Complete 8-fold Symmetry]
For real basis functions, ERIs possess 8-fold permutation symmetry:
\begin{enumerate}
    \item $\eri{\mu}{\nu}{\lambda}{\sigma} = \eri{\nu}{\mu}{\lambda}{\sigma}$
          \quad (swap in bra) --- \textbf{this is the one tested}
    \item $\eri{\mu}{\nu}{\lambda}{\sigma} = \eri{\mu}{\nu}{\sigma}{\lambda}$
          \quad (swap in ket)
    \item $\eri{\mu}{\nu}{\lambda}{\sigma} = \eri{\lambda}{\sigma}{\mu}{\nu}$
          \quad (swap bra $\leftrightarrow$ ket)
    \item All combinations of the above generate 8 equivalent orderings.
\end{enumerate}

The code also tests:
\begin{itemize}
    \item \texttt{eri[0,1,0,1]} $=$ \texttt{eri[0,1,1,0]}:
          swap $\lambda \leftrightarrow \sigma$ (symmetry 2)
    \item \texttt{eri[0,1,1,0]} $=$ \texttt{eri[1,0,0,1]}:
          combination of symmetries
\end{itemize}
\end{keyformulabox}

\textbf{Numerical verification for \ce{H2}/STO-3G:}
\begin{align}
    \texttt{eri[0,1,0,1]} &= 0.29759055094924614, \\
    \texttt{eri[1,0,0,1]} &= 0.29759055094924614, \\
    \texttt{eri[0,1,1,0]} &= 0.29759055094924614.
\end{align}
All three values are identical to machine precision, confirming the symmetries.
\end{checkpointAnswer}

%-------------------------------------------------------------------------------
\subsection{Checkpoint 1.3: Factor of 1/2 in Energy Reconstruction}
\label{sec:cp13}

\begin{checkpointAnswer}[Section 1.7 -- Energy Expression]
\textbf{Question:} PySCF's \texttt{get\_veff()} returns $\Jmat - \tfrac{1}{2}\Kmat$ for RHF.
In the energy reconstruction, we compute
$E_{\text{elec}} = \tr{\Pmat\Hcore} + \tfrac{1}{2}\tr{\Pmat \cdot \texttt{vhf}}$.
Why is there a factor of $\tfrac{1}{2}$ in front of the \texttt{vhf} term?

\textbf{Answer:}
This factor of $\tfrac{1}{2}$ has a deep physical origin related to avoiding
\textbf{double-counting} of electron-electron interactions.

\textbf{The problem of double-counting:}

The Coulomb operator $\Jmat$ represents the electrostatic interaction of an
electron with the total electron density $\rho(\mathbf{r})$:
\begin{equation}
    J_{\mu\nu} = \sum_{\lambda\sigma} \eri{\mu}{\nu}{\lambda}{\sigma} P_{\lambda\sigma}.
\end{equation}

If we computed the two-electron energy as $\tr{\Pmat\Jmat}$, we would count
each electron pair $(i,j)$ twice: once when electron $i$ interacts with the
density from electron $j$, and again when electron $j$ interacts with the
density from electron $i$. Therefore:
\begin{equation}
    E_J = \frac{1}{2} \tr{\Pmat\Jmat} = \frac{1}{2} \sum_{\mu\nu\lambda\sigma}
    P_{\mu\nu} \eri{\mu}{\nu}{\lambda}{\sigma} P_{\lambda\sigma}.
\end{equation}

The same reasoning applies to the exchange term $\Kmat$.

\textbf{Derivation of the energy expression:}

The RHF electronic energy can be written as:
\begin{equation}
    E_{\text{elec}} = \tr{\Pmat\Hcore} + \frac{1}{2}\tr{\Pmat\Jmat}
                    - \frac{1}{4}\tr{\Pmat\Kmat},
\end{equation}
where the $\frac{1}{4}$ for exchange comes from $\frac{1}{2}$ (double-counting)
times $\frac{1}{2}$ (the RHF exchange coefficient).

Grouping terms:
\begin{equation}
    E_{\text{elec}} = \tr{\Pmat\Hcore}
                    + \frac{1}{2}\tr{\Pmat\left(\Jmat - \frac{1}{2}\Kmat\right)}
                    = \tr{\Pmat\Hcore} + \frac{1}{2}\tr{\Pmat \cdot \texttt{vhf}}.
\end{equation}

\begin{keyformulabox}[Alternative Formulation]
The energy can also be written as:
\begin{equation}
    E_{\text{elec}} = \frac{1}{2}\tr{\Pmat(\Hcore + \Fmat)},
\end{equation}
where $\Fmat = \Hcore + \Jmat - \frac{1}{2}\Kmat = \Hcore + \texttt{vhf}$.

This is equivalent because:
\begin{align}
    \frac{1}{2}\tr{\Pmat(\Hcore + \Fmat)}
    &= \frac{1}{2}\tr{\Pmat\Hcore} + \frac{1}{2}\tr{\Pmat\Fmat} \\
    &= \frac{1}{2}\tr{\Pmat\Hcore} + \frac{1}{2}\tr{\Pmat\Hcore}
       + \frac{1}{2}\tr{\Pmat(\Jmat - \tfrac{1}{2}\Kmat)} \\
    &= \tr{\Pmat\Hcore} + \frac{1}{2}\tr{\Pmat \cdot \texttt{vhf}}.
\end{align}
\end{keyformulabox}

\textbf{Physical interpretation:}

The factor of $\frac{1}{2}$ ensures we count each unique electron pair
\emph{exactly once}. This is analogous to computing the electrostatic
self-energy of a charge distribution:
\begin{equation}
    E_{\text{Coulomb}} = \frac{1}{2} \iint \frac{\rho(\mathbf{r})\rho(\mathbf{r}')}{|\mathbf{r}-\mathbf{r}'|}\, d\mathbf{r}\, d\mathbf{r}',
\end{equation}
where the $\frac{1}{2}$ prevents double-counting of the $(i,j)$ and $(j,i)$ contributions.

\textbf{Numerical verification for \ce{H2}/STO-3G:}

Computing energy components:
\begin{align}
    \tr{\Pmat\Hcore} &= -2.5066 \text{ E}_{\text{h}}, \\
    \tfrac{1}{2}\tr{\Pmat \cdot \texttt{vhf}} &= 0.6748 \text{ E}_{\text{h}}, \\
    E_{\text{elec}} &= -2.5066 + 0.6748 = -1.8319 \text{ E}_{\text{h}}, \\
    E_{\text{nuc}} &= 0.7151 \text{ E}_{\text{h}}, \\
    E_{\text{tot}} &= -1.8319 + 0.7151 = -1.1168 \text{ E}_{\text{h}}.
\end{align}

This matches the PySCF RHF energy to machine precision, confirming the
correct treatment of the $\frac{1}{2}$ factor.
\end{checkpointAnswer}

% =============================================================================
% SECTION 2: LAB 1A SOLUTIONS
% =============================================================================

\section{Lab 1A: AO Integral Inventory and Sanity Checks}

This section provides the complete expected output and analysis for Lab 1A,
using \ce{H2} in the STO-3G basis at a bond length of 0.74~\AA.

\subsection{System Specification}

\begin{center}
\begin{tabular}{ll}
\toprule
Parameter & Value \\
\midrule
Molecule & \ce{H2} \\
Geometry & H at $(0,0,0)$; H at $(0,0,0.74)$ \AA \\
Basis set & STO-3G (minimal basis) \\
Number of electrons & 2 \\
Number of AOs & 2 (one $1s$ function per H) \\
\bottomrule
\end{tabular}
\end{center}

\subsection{One-Electron Integrals}

\subsubsection{Overlap Matrix $\Smat$}

\begin{equation}
    \Smat = \begin{pmatrix}
        1.00000000 & 0.65987312 \\
        0.65987312 & 1.00000000
    \end{pmatrix}
\end{equation}

\textbf{Observations:}
\begin{itemize}
    \item Diagonal elements are 1.0 (normalized basis functions)
    \item Off-diagonal element $S_{12} = 0.660$ indicates significant overlap
          between the two $1s$ functions at 0.74~\AA{} separation
    \item Matrix is symmetric: $S_{\mu\nu} = S_{\nu\mu}$
\end{itemize}

\subsubsection{Kinetic Energy Matrix $\Tmat$}

\begin{equation}
    \Tmat = \begin{pmatrix}
        0.76003188 & 0.23696027 \\
        0.23696027 & 0.76003188
    \end{pmatrix}
\end{equation}

\textbf{Observations:}
\begin{itemize}
    \item Diagonal elements $T_{11} = T_{22} = 0.760$ E$_{\text{h}}$
          represent kinetic energy of each $1s$ orbital
    \item Off-diagonal elements are positive (kinetic coupling)
    \item Matrix is symmetric
\end{itemize}

\subsubsection{Nuclear Attraction Matrix $\Vmat$}

\begin{equation}
    \Vmat = \begin{pmatrix}
        -1.88099134 & -1.19633604 \\
        -1.19633604 & -1.88099134
    \end{pmatrix}
\end{equation}

\textbf{Observations:}
\begin{itemize}
    \item All elements are \textbf{negative} (attractive potential)
    \item Diagonal elements include attraction to both nuclei
    \item Matrix is symmetric
\end{itemize}

\subsubsection{Core Hamiltonian $\Hcore = \Tmat + \Vmat$}

\begin{equation}
    \Hcore = \begin{pmatrix}
        -1.12095946 & -0.95937577 \\
        -0.95937577 & -1.12095946
    \end{pmatrix}
\end{equation}

\textbf{Observations:}
\begin{itemize}
    \item Net negative values indicate nuclear attraction dominates kinetic energy
    \item This is expected for a bound system
\end{itemize}

\subsection{Two-Electron Integrals (ERIs)}

\subsubsection{ERI Tensor Properties}

\begin{center}
\begin{tabular}{ll}
\toprule
Property & Value \\
\midrule
Shape & $(2, 2, 2, 2)$ \\
Total elements & $2^4 = 16$ \\
Memory (s1, full) & 128 bytes \\
Unique elements (8-fold symmetry) & 3 \\
\bottomrule
\end{tabular}
\end{center}

\subsubsection{Representative ERI Values}

\begin{center}
\begin{tabular}{lll}
\toprule
Element & Value (E$_{\text{h}}$) & Physical Meaning \\
\midrule
$\eri{0}{0}{0}{0}$ & 0.7746059 & On-site Coulomb (H$_1$) \\
$\eri{1}{1}{1}{1}$ & 0.7746059 & On-site Coulomb (H$_2$) \\
$\eri{0}{1}{0}{1}$ & 0.2976 & Two-center Coulomb \\
$\eri{0}{0}{1}{1}$ & 0.5697 & Two-center Coulomb (different) \\
\bottomrule
\end{tabular}
\end{center}

\subsubsection{Symmetry Verification}

The 8-fold symmetry is verified numerically:
\begin{align}
    \eri{0}{1}{0}{1} &= \eri{1}{0}{0}{1} = 0.29759055 \quad
        \text{(swap $\mu \leftrightarrow \nu$)}, \\
    \eri{0}{1}{0}{1} &= \eri{0}{1}{1}{0} = 0.29759055 \quad
        \text{(swap $\lambda \leftrightarrow \sigma$)}, \\
    \eri{0}{1}{1}{0} &= \eri{1}{0}{0}{1} = 0.29759055 \quad
        \text{(swap bra $\leftrightarrow$ ket)}.
\end{align}

\subsection{RHF Calculation Results}

\subsubsection{Energy Components}

\begin{center}
\begin{tabular}{lrl}
\toprule
Quantity & Value & Units \\
\midrule
One-electron energy $\tr{\Pmat\Hcore}$ & $-2.5066$ & E$_{\text{h}}$ \\
Two-electron energy $\frac{1}{2}\tr{\Pmat \cdot \texttt{vhf}}$ & $+0.6748$ & E$_{\text{h}}$ \\
Electronic energy $E_{\text{elec}}$ & $-1.8319$ & E$_{\text{h}}$ \\
Nuclear repulsion $E_{\text{nuc}}$ & $+0.7151$ & E$_{\text{h}}$ \\
\midrule
\textbf{Total energy} $E_{\text{tot}}$ & $\mathbf{-1.1168}$ & E$_{\text{h}}$ \\
\bottomrule
\end{tabular}
\end{center}

\subsubsection{Density Matrix $\Pmat$}

\begin{equation}
    \Pmat = \begin{pmatrix}
        0.60245569 & 0.60245569 \\
        0.60245569 & 0.60245569
    \end{pmatrix}
\end{equation}

\textbf{Observations:}
\begin{itemize}
    \item All elements are equal (symmetric bonding MO contributes equally
          to both AOs)
    \item This reflects the equal sharing of electrons in a homonuclear diatomic
\end{itemize}

\subsubsection{MO Coefficients and Energies}

\begin{equation}
    \Cmat = \begin{pmatrix}
        0.5488 & 1.2125 \\
        0.5488 & -1.2125
    \end{pmatrix}, \qquad
    \boldsymbol{\varepsilon} = \begin{pmatrix}
        -0.5786 \\
        0.6711
    \end{pmatrix} \text{ E}_{\text{h}}
\end{equation}

\textbf{Interpretation:}
\begin{itemize}
    \item Column 1: Bonding MO ($\sigma_g$) with equal coefficients
          (symmetric combination)
    \item Column 2: Antibonding MO ($\sigma_u^*$) with opposite signs
          (antisymmetric combination)
    \item HOMO-LUMO gap: $0.6711 - (-0.5786) = 1.250$ E$_{\text{h}}$
          $\approx 34$ eV
\end{itemize}

\subsubsection{Electron Count Verification}

\begin{equation}
    N_e = \tr{\Pmat\Smat} = 0.6025 \times 1.0 + 0.6025 \times 0.6599
        + 0.6025 \times 0.6599 + 0.6025 \times 1.0 = 2.0000
\end{equation}

The electron count is exactly 2, confirming correct construction of the
density matrix.

\subsubsection{MO Orthonormality Check}

\begin{equation}
    \Cmat\T\Smat\Cmat = \begin{pmatrix}
        1.0000 & 0.0000 \\
        0.0000 & 1.0000
    \end{pmatrix}
\end{equation}

The MOs are orthonormal under the $\Smat$ metric, as required.

\subsection{Energy Reconstruction Verification}

The electronic energy can be reconstructed from integrals:

\begin{lstlisting}[language=Python,caption={Energy reconstruction verification}]
# Method 1: Using get_veff()
vhf = mf.get_veff(mol, dm)  # Returns J - 0.5*K for RHF
E_elec = np.einsum("ij,ji->", dm, h) + 0.5 * np.einsum("ij,ji->", dm, vhf)
E_tot_1 = E_elec + mol.energy_nuc()

# Method 2: Using explicit J and K
J = np.einsum('ijkl,kl->ij', eri, dm)
K = np.einsum('ikjl,kl->ij', eri, dm)
F = h + J - 0.5*K
E_elec = 0.5 * np.einsum("ij,ji->", dm, h + F)
E_tot_2 = E_elec + mol.energy_nuc()

# Both should give -1.1167593073964255 Eh
\end{lstlisting}

Both methods yield $E_{\text{tot}} = -1.1167593073964255$ E$_{\text{h}}$,
matching the SCF result to machine precision (difference $< 10^{-15}$ E$_{\text{h}}$).

\subsection{Memory Scaling Analysis}

\begin{center}
\begin{tabular}{lcccc}
\toprule
Storage Mode & Elements & Memory & Reduction Factor \\
\midrule
\texttt{aosym="s1"} (full) & $N^4 = 16$ & 128 bytes & 1.0 \\
\texttt{aosym="s8"} (packed) & 3 & 24 bytes & 5.3$\times$ \\
\bottomrule
\end{tabular}
\end{center}

For larger systems, the 8-fold symmetry becomes increasingly important:
\begin{itemize}
    \item Water/STO-3G: $N = 7$, reduction from 2401 to $\sim$300 unique elements
    \item Benzene/cc-pVDZ: $N \approx 114$, reduction from $1.7 \times 10^8$
          to $\sim 2 \times 10^7$ unique elements
\end{itemize}

% =============================================================================
% SECTION 3: COMMON ERRORS AND DEBUGGING
% =============================================================================

\section{Common Errors and Debugging Guide}

\subsection{Factor of 2 Errors in RHF}

The most common source of errors in RHF calculations involves factors of 2:

\begin{center}
\begin{tabular}{lll}
\toprule
Quantity & Correct Expression & Common Error \\
\midrule
Density matrix & $P_{\mu\nu} = 2\sum_i C_{\mu i}C_{\nu i}$ & Missing factor of 2 \\
Fock matrix & $\Fmat = \Hcore + \Jmat - \frac{1}{2}\Kmat$ & Using $-\Kmat$ instead \\
Electron count & $N_e = \tr{\Pmat\Smat}$ & Using $\tr{\Pmat}$ \\
\bottomrule
\end{tabular}
\end{center}

\subsection{Chemist's vs Physicist's Notation}

PySCF uses \textbf{chemist's notation}:
\begin{equation}
    \texttt{eri[}\mu,\nu,\lambda,\sigma\texttt{]}
    \equiv \eri{\mu}{\nu}{\lambda}{\sigma}
    = \iint \chi_\mu(1)\chi_\nu(1) \frac{1}{r_{12}} \chi_\lambda(2)\chi_\sigma(2).
\end{equation}

\textbf{Physicist's notation} (NOT used in PySCF):
\begin{equation}
    \langle \mu\lambda | \nu\sigma \rangle
    = \iint \chi_\mu^*(1)\chi_\lambda^*(2) \frac{1}{r_{12}} \chi_\nu(1)\chi_\sigma(2).
\end{equation}

The index ordering differs! When translating from physics textbooks:
\begin{equation}
    \langle \mu\lambda | \nu\sigma \rangle_{\text{phys}}
    = \eri{\mu}{\nu}{\lambda}{\sigma}_{\text{chem}}.
\end{equation}

\subsection{Unit Errors}

\begin{center}
\begin{tabular}{lll}
\toprule
Common Mistake & Symptom & Fix \\
\midrule
Geometry in wrong units & Energy off by factor & Check \texttt{unit=} setting \\
Mixing \AA{} and Bohr & Bond lengths wrong & PySCF defaults to \AA{} \\
\bottomrule
\end{tabular}
\end{center}

% =============================================================================
% APPENDIX: NUMERICAL VALUES REFERENCE
% =============================================================================

\section{Reference: Complete Numerical Output}

For reproducibility, here is the complete expected output from Lab 1A:

\begin{lstlisting}[language={},basicstyle=\ttfamily\footnotesize,frame=single]
Number of AOs (nao): 2
S symmetric: True
h symmetric: True

--- ERI Symmetry Checks ---
(01|01) = (10|01): True
(01|01) = (01|10): True
(01|10) = (10|01): True

ERI shape: (2, 2, 2, 2)
ERI memory (MB): 0.0001220703125

ERI (s8) shape: (3,)
ERI (s8) memory (MB): 2.288818359375e-05

RHF total energy (Eh): -1.1167593073964255
Electron count Tr[PS]: 2.0000000000000004

E_tot rebuilt (Eh): -1.1167593073964255
Difference (Eh): 0.0
\end{lstlisting}

%%%%%%%%%%%%%%%%%%%%%%%%%%%%%%%%%%%%%%%%%%%%%%%%%%%%%%%%%%%%%%%%%%%%%%%%%%%%%%%%
\section{Exercise Answer Keys}
%%%%%%%%%%%%%%%%%%%%%%%%%%%%%%%%%%%%%%%%%%%%%%%%%%%%%%%%%%%%%%%%%%%%%%%%%%%%%%%%

Brief answers for the end-of-chapter exercises (Section 1.9).

%-------------------------------------------------------------------------------
\subsection{Exercise 1.1: Identifying Hamiltonian Terms [Core]}

\begin{keyInsight}[Operator Identification]
From the electronic Hamiltonian (Eq.~1.3 in the notes):
\[
\Hop_e = \sum_{i=1}^{N_e}\left(-\frac{1}{2}\nabla_i^2 - \sum_{A=1}^{N_n}\frac{Z_A}{r_{iA}}\right)
+ \sum_{i<j}^{N_e}\frac{1}{r_{ij}}
\]

\textbf{(a)} The kinetic energy matrix $\Tmat$ comes from the $-\frac{1}{2}\nabla^2$ operators.

\textbf{(b)} The nuclear attraction matrix $\Vmat$ comes from the $-Z_A/r_{iA}$ terms.

\textbf{(c)} The ERIs arise from the electron--electron repulsion $r_{ij}^{-1}$.

\textbf{(d)} Operator counts for $N_e$ electrons and $M$ nuclei:
\begin{itemize}
    \item Kinetic energy operators: $N_e$ (one per electron)
    \item Nuclear attraction operators: $N_e \times M$ (each electron with each nucleus)
    \item Electron--electron operators: $\binom{N_e}{2} = \frac{N_e(N_e-1)}{2}$ (unique pairs)
\end{itemize}
\end{keyInsight}

%-------------------------------------------------------------------------------
\subsection{Exercise 1.2: Tracing the Computational Pipeline [Core]}

\textbf{Expected results for \ce{H2O}/STO-3G:}

\begin{center}
\begin{tabular}{ll}
\toprule
\textbf{Quantity} & \textbf{Value} \\
\midrule
Bond length $r_{\ce{OH}}$ & 0.96 \AA \\
Bond angle $\angle\ce{HOH}$ & 104.5$^\circ$ \\
$E_{\text{nuc}}$ & 9.08804 E$_\text{h}$ \\
Number of AOs ($N$) & 7 \\
SCF iterations & 10--12 (with DIIS) \\
Default threshold & $10^{-9}$ E$_\text{h}$ \\
$E_{\text{elec}}$ & $-84.03$ E$_\text{h}$ \\
$E_{\text{tot}}$ & $-74.94$ E$_\text{h}$ \\
\bottomrule
\end{tabular}
\end{center}

\begin{keyInsight}[Basis Function Count]
STO-3G assigns:
\begin{itemize}
    \item Oxygen: 1 (1s) + 1 (2s) + 3 (2p) = 5 functions
    \item Each hydrogen: 1 (1s) = 1 function
    \item Total: $5 + 1 + 1 = 7$ AO basis functions
\end{itemize}
The matrices $\Smat$, $\Hcore$, $\Fmat$, $\Pmat$ are all $7 \times 7$.
\end{keyInsight}

%-------------------------------------------------------------------------------
\subsection{Exercise 1.3: Scaling Snapshot [Core]}

\textbf{Example results for \ce{H2O}:}

\begin{center}
\begin{tabular}{lccc}
\toprule
\textbf{Basis} & \textbf{$N$} & \textbf{s1 (MB)} & \textbf{s8 (MB)} \\
\midrule
STO-3G & 7 & 0.018 & 0.003 \\
cc-pVDZ & 24 & 2.65 & 0.35 \\
\bottomrule
\end{tabular}
\end{center}

\begin{keyInsight}[Memory Scaling]
The ratio of s1 to s8 storage:
\[
\frac{N^4}{N^4/8} = 8 \quad \text{(in the limit of large $N$)}
\]

For small $N$, the ratio is less than 8 because indexing overhead matters.
Observed ratios: STO-3G $\approx 6$, cc-pVDZ $\approx 7.6$.
As $N$ grows, the ratio approaches 8.
\end{keyInsight}

%-------------------------------------------------------------------------------
\subsection{Exercise 1.4: Electron Count and Orthonormality [Core]}

\begin{keyInsight}[Role of the Overlap Matrix]
The overlap matrix $\Smat$ appears in both formulas because it defines the metric
(inner product) in the nonorthonormal AO basis:
\[
\langle u, v \rangle_S = u^\top \Smat v
\]

\textbf{MO Orthonormality:} $\Cmat^\top \Smat \Cmat = \mat{I}$

The MOs are orthonormal \emph{under the $\Smat$-metric}, not the Euclidean metric.
This means $\langle \phi_p | \phi_q \rangle = \delta_{pq}$ when computed with proper overlap.

\textbf{Electron Count:} $\tr{\Pmat\Smat} = N_e$

Without $\Smat$, we would overcount electrons in regions where AOs overlap.
The overlap matrix ``corrects'' for the redundancy in non-orthogonal descriptions.

\textbf{Physical interpretation:} $\Smat$ acts as a metric tensor that properly weights
the contribution of each AO according to its overlap with others. In an orthonormal
basis (e.g., MO basis), $\Smat = \mat{I}$ and these formulas reduce to simple traces.
\end{keyInsight}

%-------------------------------------------------------------------------------
\subsection{Exercise 1.5: ERI Symmetry Spot Check [Advanced]}

\begin{keyInsight}[Physical Origins of ERI Symmetry]
The 8-fold ERI symmetry arises from three independent physical properties:

\textbf{1. Commutativity of multiplication} (real basis):
\[
\chi_\mu(\rvec_1)\chi_\nu(\rvec_1) = \chi_\nu(\rvec_1)\chi_\mu(\rvec_1)
\]
This gives: $\eri{\mu}{\nu}{\lambda}{\sigma} = \eri{\nu}{\mu}{\lambda}{\sigma}$
and $\eri{\mu}{\nu}{\lambda}{\sigma} = \eri{\mu}{\nu}{\sigma}{\lambda}$

\textbf{2. Symmetry of Coulomb operator}:
\[
\frac{1}{r_{12}} = \frac{1}{r_{21}}
\]
This gives: $\eri{\mu}{\nu}{\lambda}{\sigma} = \eri{\lambda}{\sigma}{\mu}{\nu}$

\textbf{3. Real basis functions}:
All basis functions are real, so complex conjugation has no effect.

Combining these three 2-fold symmetries: $2 \times 2 \times 2 = 8$-fold total.

\textbf{Numerical verification:} For any randomly chosen $(\mu,\nu,\lambda,\sigma)$,
all 8 permutations should agree to machine precision ($< 10^{-14}$).
\end{keyInsight}

%-------------------------------------------------------------------------------
\subsection{Exercise 1.6: Back-of-Envelope ERI Scaling [Advanced]}

\begin{keyInsight}[Benzene cc-pVDZ Estimates]
\textbf{(a) Basis function count:}
\begin{align*}
N &\approx 6 \times 14 + 6 \times 5 = 84 + 30 = 114 \text{ functions}
\end{align*}

\textbf{(b) Full ERI memory:}
\begin{align*}
\text{Memory} &= N^4 \times 8 \text{ bytes} \\
&= 114^4 \times 8 = 1.35 \times 10^9 \text{ bytes} \approx 1.3 \text{ GB}
\end{align*}

\textbf{(c) Maximum $N$ for 16 GB RAM:}
\begin{align*}
N^4 \times 8 &\leq 16 \times 10^9 \\
N &\leq \left(\frac{16 \times 10^9}{8}\right)^{1/4} \approx 212
\end{align*}

\textbf{(d) Motivation for alternative methods:}
\begin{itemize}
    \item \textbf{Direct SCF}: Never store ERIs; recompute each iteration.
          Trades compute time for memory.
    \item \textbf{Density Fitting}: Approximate 4-index ERIs with 3-index quantities.
          Reduces $O(N^4)$ storage to $O(N^2 N_{\text{aux}})$.
\end{itemize}
\end{keyInsight}

%-------------------------------------------------------------------------------
\subsection{Exercise 1.7: Physical Interpretation of J and K [Advanced]}

\begin{keyInsight}[Coulomb Matrix]
\textbf{(a)} $\Jmat$ represents classical electrostatic repulsion:
\[
J_{\mu\nu} = \int \chi_\mu(\rvec_1)\chi_\nu(\rvec_1)
\left[\int \frac{\rho(\rvec_2)}{r_{12}} d\rvec_2\right] d\rvec_1
\]
This is the Coulomb potential from density $\rho(\rvec) = \sum_{\lambda\sigma} P_{\lambda\sigma} \chi_\lambda \chi_\sigma$.
Unlike classical point charges, quantum $\rho$ is a smooth, delocalized distribution.
\end{keyInsight}

\begin{keyInsight}[Exchange Matrix]
\textbf{(b)} $\Kmat$ arises from antisymmetry (Pauli exclusion):
\[
\Psi(1,2) = -\Psi(2,1)
\]
When two electrons have the same spin, their spatial wavefunction must be antisymmetric.
This creates an ``exchange hole''---a depletion of same-spin electron density near each electron.
There is no classical analog because classical particles are distinguishable.
\end{keyInsight}

\begin{keyInsight}[Factor of 1/2 in RHF]
\textbf{(c)} In closed-shell RHF with doubly-occupied orbitals:
\begin{itemize}
    \item Coulomb: All $N_e^2$ electron pairs contribute (including opposite spins)
    \item Exchange: Only $N_e^2/2$ same-spin pairs contribute
\end{itemize}
The factor $\frac{1}{2}$ in $\Fmat = \Hcore + \Jmat - \frac{1}{2}\Kmat$ reflects that
exchange only occurs between electrons of the same spin.
\end{keyInsight}

\begin{keyInsight}[Self-Interaction Cancellation]
\textbf{(d)} For a single doubly-occupied orbital (e.g., He $1s^2$):
\[
J_{11} = \eri{1}{1}{1}{1} = K_{11}
\]
In the energy: $J_{11} - \frac{1}{2}K_{11} - \frac{1}{2}K_{11} = J_{11} - K_{11} = 0$

This shows that HF is \textbf{self-interaction free}: an electron does not repel itself.
The exchange term exactly cancels the spurious self-Coulomb contribution.
\end{keyInsight}

%-------------------------------------------------------------------------------
\subsection{Exercise 1.8: Energy Reconstruction Without get\_veff [Challenge]}

\textbf{Key verification points:}

\begin{lstlisting}[language=Python,basicstyle=\footnotesize\ttfamily]
# For H2/STO-3G at 0.74 Angstrom
# 1. Get converged density
dm = mf.make_rdm1()

# 2. Build J matrix
J = np.einsum('mnls,ls->mn', eri, dm)

# 3. Build K matrix (note index order!)
K = np.einsum('mlns,ls->mn', eri, dm)

# 4. Fock matrix
F = h + J - 0.5*K

# 5. Electronic energy
E_elec = 0.5 * np.einsum('ij,ji->', dm, h + F)

# Expected: E_elec = -1.8319 Eh, E_tot = -1.1168 Eh
\end{lstlisting}

\begin{keyInsight}[Index Contraction Patterns]
The critical difference between $\Jmat$ and $\Kmat$:
\begin{align*}
J_{\mu\nu} &= \sum_{\lambda\sigma} \eri{\mu}{\nu}{\lambda}{\sigma} P_{\lambda\sigma}
\quad\to\quad \texttt{einsum('mnls,ls->mn', eri, dm)} \\
K_{\mu\nu} &= \sum_{\lambda\sigma} \eri{\mu}{\lambda}{\nu}{\sigma} P_{\lambda\sigma}
\quad\to\quad \texttt{einsum('mlns,ls->mn', eri, dm)}
\end{align*}
The indices being contracted ($\lambda, \sigma$) are in positions (2,3) for $\Jmat$
but positions (1,3) for $\Kmat$. This reflects the different physical origins:
$\Jmat$ is a local potential; $\Kmat$ is nonlocal (depends on orbital shape).
\end{keyInsight}

%-------------------------------------------------------------------------------
\subsection{Exercise 1.9: Debugging an SCF Calculation [Challenge]}

\begin{keyInsight}[Bug Identification]
The buggy code contains \textbf{three errors}:

\textbf{Bug 1: Missing factor of 2 in density matrix}
\begin{lstlisting}[language=Python,basicstyle=\footnotesize\ttfamily]
# Wrong:
P = np.einsum('mi,ni->mn', C_occ, C_occ)
# Correct:
P = 2 * np.einsum('mi,ni->mn', C_occ, C_occ)
\end{lstlisting}
For RHF, each spatial orbital is doubly occupied. The density matrix must include
the factor of 2: $P_{\mu\nu} = 2\sum_i C_{\mu i} C_{\nu i}$.

\textbf{Bug 2: Missing factor of 1/2 on exchange in Fock matrix}
\begin{lstlisting}[language=Python,basicstyle=\footnotesize\ttfamily]
# Wrong:
F = h + J - K
# Correct:
F = h + J - 0.5*K
\end{lstlisting}
Exchange only occurs between same-spin electrons. With double occupation,
only half the electron pairs are same-spin.

\textbf{Bug 3: Missing factor of 1/2 in energy formula}
\begin{lstlisting}[language=Python,basicstyle=\footnotesize\ttfamily]
# Wrong:
E_elec = np.trace(P @ (h + F))
# Correct:
E_elec = 0.5 * np.trace(P @ (h + F))
\end{lstlisting}
The factor of $\frac{1}{2}$ prevents double-counting electron pairs.
When summing over all $\mu,\nu$, each pair $(i,j)$ is counted as both
$(i,j)$ and $(j,i)$.
\end{keyInsight}

\begin{keyInsight}[Debugging Strategy]
To catch these errors systematically:
\begin{enumerate}
    \item Check $\tr{\Pmat\Smat} = N_e$ (catches Bug 1)
    \item Compare $\Fmat$ against \texttt{mf.get\_fock()} (catches Bug 2)
    \item Compare $E_{\text{elec}}$ against \texttt{mf.energy\_elec()[0]} (catches Bug 3)
\end{enumerate}
\end{keyInsight}

\vspace{2em}

\begin{center}
\rule{0.5\textwidth}{0.4pt}\\[0.5em]
{\small\itshape End of Chapter 1 Solutions}
\end{center}

\end{document}
